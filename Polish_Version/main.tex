\documentclass{SGGW-thesis}

\INZYNIERSKAtrue
\WZIMtrue

\title{Zastosowanie metod uczenia maszynowego do stworzenia sztucznej inteligencji w turowych grach RPG}
\Etitle{The use of machine learning methods to create artificial intelligence in turn-based RPG games}
\author{Rafał Kuligowski}
\date{2023}
\album{205835}
\thesis{Praca dyplomowa na kierunku:}
\course{Informatyka}
\promotor{dra Marka Karwańskiego}
\pworkplace{Instytut Informatyki Technicznej\\Katedra Zastosowań Matematyki}

\usepackage{hyperref}

\begin{document}
\maketitle
\statementpage
\abstractpage
{Zastosowanie metod uczenia maszynowego do stworzenia sztucznej inteligencji w turowych grach RPG}
{Tematem pracy była implementacja }
{uczenie maszynowe, sztuczna inteligencja, gry RPG, uczenie przez wzmacnianie, turowe systemy walki}
{The use of machine learning methods to create artificial intelligence in turn-based RPG games}
{TO CHANGE: The subject of this study was to implement a \LaTeX\ class that allows for text formatting according to the guidelines imposed by the University. The work consists of two
main parts. The first one describes the most important aspects of the implementation. The second part focuses on how to use the class by people writing the theses.}
{machine learning, artificial intelligence, RPG games, reinforcement learning, turn-based battle system}


{
  % Spis treści może być złożony z pojedynczą interlinią, np. jeśli jedna linia wychodzi na następną stronę.
  % W przeciwnym razie spis treści wstawić bez powyższego rozkazu i klamry.
  \doublespacing
  \tableofcontents
}

\startchapterfromoddpage % niezależnie od długości spisu treści pierwszy rozdział zacznie się na nieparzystej stronie

\chapter{Wstęp}
Sztuczna Inteligencja w przeciągu ostatnich lat objęła wiele dziedzin życia i jest ciągle dynamicznie rozwijana. Jej zastosowanie jest bardzo 
rozległe, a jedną z branż o dużych perspektywach rozwoju tej technologii jest przemysł rozrywkowy w który wchodzą gry wideo. W grach można zastosować 
tą technologię do chociażby generowania misji i zadań dla graczy aby nie były one powtarzalne, stworzenia mądrego systemu automatycznej walki dzięki
któremu gracz nie lubiący tego elementu w grze może go pominąć, czy też do stworzenia sztucznej inteligencji dla przeciwników gracza. Ostatnie z tych 
przykładowych zastosowań zostanie zgodnie z tematem pracy omówione pod względem teoretycznym i praktycznym.


\section{Cel i zakres pracy}
Praca ma na celu zbadanie możliwości implementacji sztucznej inteligencji dla przeciwnków w grach wideo z gatunku RPG\footnote{RPG (ang. Role Playing Games) 
- gry w których gracz wciela się w rolę postaci występujących w fikcyjnym świecie. Gracze ponoszą wszelkie konsekwencje swoich akcji jako postać w świecie gry.}
z turowym systemem walki. Wynikowo otrzymamy 2 aplikacje: trenującą model oraz testującą model w postaci symulatora walki. Treść pracy będą stanowić opisy technologii
użytych w pracy wraz z alternatywami, opis implementacji rozwiązania na podstawie wybranych wcześniej technologii, instrukcję obsługi napisanej aplikacji oraz
wnioski na temat implementacji takiego rozwiązania w szerszym zakresie.


\chapter{Technologie}
W tym rozdziale zostaną szczegółowo omówione komponenty potrzebne do implementacji takowej sztucznej inteligencji. Są to: 
\begin{itemize}
  \item{Rodzaj uczenia maszynowego}
  \item{Wykorzystany algorytm do uczenia maszynowego}
  \item{Silnik do stworzenia gry}
  \item{System walki (w przypadku tej pracy - system oparty o tury)}
\end{itemize}


\section{Rodzaje uczenia maszynowego}
W uczeniu maszynowym można rozgraniczyć 3 główne rodzaje (w oparciu o~\cite{MachineLearningTypes})



\begin{thebibliography}{9}
  \bibitem{MachineLearningTypes}
  Shagan Sah,
  \textit{Machine Learning: A Review of Learning Types}
  (2020 r.)

  \bibitem{wymagania}
  \textit{Zarządzenie nr 34 Rektora Szkoły Głównej Gospodarstwa Wiejskiego w Warszawie z dnia 01 czerwca 2016 r.\ w~ sprawie wprowadzenia ,,Wytycznych dotyczących
przygotowywania prac dyplomowych w~Szkole Głównej Gospodarstwa Wiejskiego w Warszawie''}, Załączniki 1 i~2
  \url{https://www.sggw.pl/dla-studentow/informacje-formalno-prawne/dokumenty-do-pobrania}
  $\rightarrow$ Praca dyplomowa (dostęp: 04.01.2017)
\end{thebibliography}

\beforelastpage

\end{document}